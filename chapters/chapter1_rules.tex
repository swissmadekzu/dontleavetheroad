\chapter{Système de jeu}
\newpage

\section{Qu'est-ce que le jeu de rôle}

Selon l'article \href{http://www.ffjdr.org/quest-ce-que-le-jeu-de-role/}{Qu’est-ce que le jeu de rôle ?} de la Fédération Française de Jeu de Rôle.

Le jeu de rôle est une activité de loisir et de divertissement qui consiste à raconter une histoire à plusieurs. Installés autour d’une table, les joueurs interprètent le rôle d’un personnage au sein d’un univers fictif. Chacun décrit comment son personnage interagit avec le monde et les personnages qui l’entourent.

\subsection*{Comment ça marche ?}

Le jeu réside dans les échanges entre les participants. Le Meneur de Jeu ou MJ raconte l’histoire. Il plante le décor du jeu, met en place des défis à surmonter, arbitre les situations et interprète les rôles secondaires ; il assure la cohérence de l’histoire que vont vivre les joueurs. Les autres joueurs interprètent leur personnage, le plus souvent oralement, à la première personne. Ils le font agir dans le monde décrit par le MJ afin d’accomplir des objectifs communs à tout le groupe. Il existe également des jeux sans MJ, dans ce cas le rôle du MJ est réparti différemment entre les joueurs.

\subsection*{À quoi ça ressemble ?}

Un jeu de rôle a souvent la forme d’un livre qui contient tout le nécessaire pour jouer :

\begin{itemize}
\item Un univers de jeu. Il peut s’agir d’un univers réaliste, fantastique, futuriste. L’univers peut être original (créé pour le jeu), ou inspiré d’un livre, d’un film…
\item Des règles du jeu, également appelées “système de jeu”. Elles quantifient les forces et les faiblesses des personnages, ainsi que la difficulté des diverses actions réalisables dans le jeu. Elles aident ainsi le MJ à décider si les actions tentées par les joueurs réussissent ou échouent au sein de l’histoire. La plupart des règles du jeu utilisent des dés pour instaurer une part de hasard.
\item Un ou plusieurs scénarios. Le scénario est un canevas présentant des lieux, des personnages et des situations. Il fonctionne un peu comme le script d’un film, sauf que rien n’est totalement prévu à l’avance : l’histoire va évoluer en fonction des choix que vont faire les joueurs, et des directions qu’ils vont vouloir explorer. Le talent du MJ réside dans sa capacité à cadrer l’histoire grâce au scénario, tout en improvisant les situations qui ne sont pas prévues par ce dernier.
\end{itemize}

Certains jeux de rôle proposent des extensions, des livres additionnels qui développent certains aspects de l’univers, qui proposent de nouveaux scénarios ou qui ajoutent des règles supplémentaires. Dans ce cas, le jeu de rôle n’est plus seulement un livre unique : il devient une gamme.

\subsection*{Comment jouer ?}

Pour jouer, il faut avant tout réunir un groupe de joueurs. De nombreuses associations de jeu de rôle existent. Il est également possible de jouer par Internet, en jeu de rôle virtuel, ou de faire une recherche de joueurs sur le forum de la FFJdR.

L’un des joueurs possède un livre de jeu de rôle et il a pris soin de bien lire les règles du jeu avant la partie. Il a également prévu un scénario, qu’il a choisi tout prêt ou qu’il a créé lui-même. Ce joueur, c’est le MJ. C’est lui qui va se charger d’expliquer les règles du jeu aux autres joueurs.

Sous la houlette du MJ, les autres joueurs remplissent leur fiche de personnage (préalablement photocopiée) afin de donner vie à leur personnage. Ils se seront munis de dés adaptés au jeu joué, d’un crayon, d’une gomme, et de papier pour prendre des notes. Une fois toutes ces conditions réunies, et la fiche de personnage complétée, le MJ commence à raconter l’histoire. C’est parti !

\subsection*{Quel est l’objectif de la partie ?}

L’objectif des joueurs est de mettre en scène verbalement et de façon collaborative, des situations aussi riches et intéressantes que possible. Leur objectif est également d’incarner leur personnage et de défendre ses intérêts. Ils créent ainsi un récit interactif qui contribue au développement d’un univers imaginaire passionnant. Toutes les thématiques et tous les scénarios sont possibles, la seule limite étant l’imagination des participants.

L’objectif des personnages varie selon le scénario et selon les choix des joueurs lors de la création du personnage, au début de la partie : retrouver leur bien-aimé(e), sauver le monde, devenir riches et célèbres, se venger, semer la mort et la dévastation… La plupart du temps, le joueur se prendra au jeu et les objectifs de son personnage deviendront les siens. Parfois, il sera content que son personnage ait échoué, car cela créera une histoire intéressante. Une bonne partie, ce n’est pas quand le personnage gagne. Une bonne partie, c’est quand les joueurs s’amusent.

Il n’y a jamais ni gagnant ni perdant dans un jeu de rôle, le seul véritable but du jeu est le plaisir que l’on trouve à y jouer. La partie s’arrête en fonction du temps disponible ou lorsque l’on arrive à la fin du récit. Cette fin est définie par l’accomplissement de l’objectif des joueurs, par la mort des personnages ou par la fin du scénario du MJ.

On peut continuer à jouer indéfiniment les mêmes personnages dans ce que l’on appelle une campagne, un peu à la manière des épisodes d’une série télévisée. On peut également jouer une partie unique appelée oneshot, dont le récit connaîtra une fin rapide.

\newpage

\section{Système de résolution}

\lettrine[lines=3]{\initfamily\textcolor{darkgreen}{L}}{e système} de jeu se veut très simple et axé sur la narration plus que sur des jets de dés à répétition. Le d12 est le principal (seul) dé utilisé. 

\subsection{Caractéristiques}

Les caractéristiques sont au nombre de 3 :

\begin{itemize}
	\item \textbf{Corps :} Pour toutes les actions physiques et la résistance aux chocs.
	\item \textbf{Esprit :} Pour les actions mentales et la résistance à la folie et à la peur.
	\item \textbf{Âme :} Pour les actions sociales et la résistance aux intimidations et mensonges divers et variés.
\end{itemize}

La valeur d'une caractéristique se situe entre 0 (amorphe ou mort) et 12 (surhumanité totale). La moyenne humaine d'une caractéristique est de 5.

\subsubsection*{Valeur et réserve de caractéristiques}

La valeur de caractéristique est (pour un nouveau personnage) le nombre de points attribués à la caractéristique.

La réserve de caractéristique est le nombre actuel de points qu'a le personnage dans sa caractéristique. Elle monte et descend lors de l'histoire.

Lorsque la réserve de caractéristique arrive à 0, le personnage n'est plus jouable. Le joueur doit alors créer un nouveau personnage. À 0 de Corps, un personnage est mort, rekt, fini quoi. À 0 d'Esprit, il est complètement fou. À 0 d'Âme, il est catatonique et ne s'exprime plus que par grognements.

\subsubsection*{Prendre des dommages dans une caractéristique}

Lorsqu'un personnage subit des dégâts physiques, ceux-ci sont directement impactés dans sa caractéristique de Corps.

Lorsqu'il subit une grande peur ou des émotions mentales divergentes (par exemple, en étant attiré et révulsé à la fois par la bête qu'il voit), cela crée des dégâts mentaux qui sont impactés dans la caractéristique d'Esprit

Lorsqu'il est intimidé fortement ou encore ridiculisé en public, c'est la caractéristique d'Âme qui est touchée.

\subsubsection*{Regagner sa réserve de caractéristique}

Lorsqu'un personnage se repose dans un lieu (relativement) sûr pendant quatre heures, il regagne un point dans une de ses réserves de caractéristiques. S'il choisit de se reposer huit heures, il aura donc deux points à répartir entre deux caractéristiques, ou deux points dans une seule, à choix. Trois points pour douze heures, etc... On parle ici de repos, cela peut donc être trainer dans le bar d'une communauté (si le bar est assez sûr, je déconseille à tous le bouge des trafiquants de cerveaux), dormir, ou encore entretenir son équipement.

À la fin d'un \textbf{Scénario}, un personnage remonte toutes ses réserves à la valeur de caractéristique.

\subsection{Aspects}

Un personnage est défini par trois aspects :

\begin{itemize}
	\item \textbf{Concept :} Le concept du personnage définit qui il est, quelle est sa spécialité, et quel peut être son trait de caractère le plus marquant. On définira par exemple un personnage comme \emph{Bûcheron acariâtre} ou encore \emph{Bourgmestre cupide}
	\item \textbf{Réconfort :} Il s'agit ici de quelque chose ou quelqu'un qui apporte un grand réconfort au personnage. Cela peut être une petite amie, un colifichet, une arme, tout (ou presque) est possible. Le réconfort permet au personnage de regagner ses points de caractéristique perdus deux fois plus vite, uniquement lorsqu'il est à proximité. Dans le cas où le personnage a perdu (définitivement ou non) sa source de réconfort, il perd cette aptitude jusqu'à la fin d'un \textbf{Cycle}. \emph{(Voir ci-dessous)} (ou jusqu'à ce qu'il le retrouve, ce qui peut donner lieu à un Scénario ou à un Cycle complet). À la fin d'un Cycle, il peut choisir un nouveau Réconfort.
	\item \textbf{Crainte :} La crainte du personnage est sa plus grande peur. Elle doit être cohérente avec l'univers. Lorsqu'il est confronté à sa crainte, le personnage doit immédiatement faire un test d'Esprit à difficulté 12. S'il réussit, il a combattu sa peur et gagne un Point de Destin. S'il échoue, il perd un point d'Esprit.
\end{itemize}

\subsection{Points de destin}

Dans des cas extrêmes, un personnage peut influencer quelque peu le destin. Pour ce faire, il bénéficie de points de destin. Un point de destin peut être dépensé pour :

\begin{itemize}
	\item Relancer un jet, quel que soit le résultat
	\item Éviter de mourir, devenir fou ou amorphe. Lorsque des dégâts devraient amener une de ses caractéristiques à 0, il peut dépenser un point de destin pour ramener sa réserve de caractéristique à 1, et être simplement inconscient. Le MJ ne devrait pas s'acharner sur le personnage, après tout, il vient de survivre à une mort certaine.
\end{itemize}

\subsection{Actions}

Lorsque le résultat d'une action n'est pas automatiquement réussi ou automatiquement échoué, on lance 1d12 auquel on ajoute la réserve de caractéristique liée, et on compare à un seuil de réussite. 

\begin{itemize}
	\item Si le résultat est supérieur au seuil, l'action est \textbf{réussie}
	\item Si le résultat est égal, l'action est \textbf{réussie}, mais avec une \textbf{complication}
	\item Si le résultat est inférieur, l'action est \textbf{échouée}
\end{itemize}

\subsubsection*{Les extrêmes du dé}

Si le résultat du dé est un 12 naturel, on dit que la réussite est critique. Cela entraîne les possibilités suivantes :

\begin{itemize}
	\item Le personnage a si bien réussi son action, que le prochain jet en relation, que ce soit le sien ou celui de l'un de ses alliés, bénéficiera d'un bonus de +3.
	\item Le personnage a visé un point sensible de son adversaire, et inflige donc deux fois plus de dommages.
	\item L'action du personnage visait à handicaper son adversaire, qui subira donc un malus de -3 à son prochain jet en relation.
\end{itemize}

Si le résultat du dé est un 1 naturel, il s'agit d'un échec critique, et ce n'est généralement pas bon... Cela entraîne les possibilités suivantes :

\begin{itemize}
	\item Non seulement le personnage a complètement raté son action, mais il s'est blessé dans la manoeuvre. Il perd un point de sa réserve de la caractéristique associée.
	\item En attaquant son adversaire, il a gravement ouvert sa garde. Lors de sa prochaine attaque, l'adversaire aura un bonus de +3.
	\item Quelque chose de très mauvais s'est produit. Le MJ peut imaginer n'importe quelle complication.
\end{itemize}

\subsubsection*{Donner de sa personne}

Lorsqu'un personnage effectue une action, il peut choisir de donner de soi-même pour réussir son action. (Cela peut être fait avant ou après l'annonce du résultat)

Lorsqu'il donne de sa personne, il dépense un point de la réserve de la caractéristique concernée, et peut alors faire augmenter le degré de réussite d'un cran. Attention, un échec critique restera un échec critique, et on ne peut pas obtenir de réussite critique de cette façon !

Dépenser un point de sa réserve peut donc :

\begin{itemize}
	\item Faire passer d'un échec à une réussite avec complication
	\item Faire passer d'une réussite avec complication en réussite
\end{itemize}

\subsection{Scénario, Cycle et Campagne}

\subsubsection*{Scénario}

Un Scénario est défini généralement par un but (aller chercher un voyageur perdu, apporter un message d'une communauté à une autre, comprendre pourquoi le bourgmestre de Trifouilli-les-oies est une ordure finie et l'amener à la raison ou à son destin, ...). Un Scénario peut se dérouler sur plusieurs séances de jeu.

À la fin d'un Scénario, les personnages :

\begin{itemize}
	\item Regagnent toutes leurs réserves de caractéristiques à hauteur de leurs valeurs.
	\item Sont heureux d'avoir survécu aux expériences de ce monde de merde...
	\item Gagnent de l'expérience en fonction de leurs actions et des répercussions que celles-ci ont eu sur le monde.
\end{itemize}

\subsubsection*{Cycle}

Un Cycle est une suite de scénarios définis par une ligne directrice (qui peut être très floue). Cela peut être un ensemble de Scénarios où l'un des personnages est impliqué par sa famille, ou bien plusieurs missions données par un même donneur d'ordres, etc...

À la fin d'un Cycle, les personnages :

\begin{itemize}
	\item Gagnent les mêmes bonus que lors d'une fin de Scénario
	\item Peuvent modifier leur Réconfort, ou choisir un nouveau Réconfort s'ils ont définitivement perdu le leur.
\end{itemize}

\subsubsection*{Campagne}

Une Campagne est une suite de Cycles qui raconte l'histoire vécue par les personnages en jeu.

À la fin d'une Campagne, les personnages partent à la retraite...

\newpage

\section{Création de personnage}

\lettrine[lines=3]{\initfamily\textcolor{darkgreen}{L}}{a création} de personnage pour Don't Leave The Road se déroule en plusieurs étapes.

\subsection{Choisir un concept de personnage}

En tout premier lieu, il est important de savoir quoi jouer. Par conséquent, la toute première chose à décider est le concept du personnage. Ce qu'il est, ce qu'il aime.

\emph{Exemple : Bob a envie de jouer une combattante farouche qui s'énerve facilement. Il va donc inscrire dans son concept : Epéiste au sale caractère.}

Un bon concept devrait de préférence donner des indications sur la spécialité et un aspect de la psychologie du personnage.

Inscrivez également le nom du personnage si vous avez déjà choisi. Autrement vous avez encore du temps pour décider durant les prochaines étapes.

\subsection{Répartir les points de caractéristique}

Un personnage dispose de trois caractéristiques : Le Corps, l'Esprit et l'Âme. Chacune de ces caractéristiques peut avoir une valeur comprise entre 2 et 8 à la création. On considèrera que 5 est une valeur moyenne pour un humain lambda.

Un personnage débutant dispose de 17 points à répartir entre ces trois caractéristiques.

\emph{Exemple : Pour sa combattante, Bob a choisi de faire un personnage équilibré, mais dont la caractéristique d'Âme est assez basse, rapport au caractère de cochon. Il répartit donc 7 points en Corps, 6 en Esprit et 4 en Âme.}

Un personnage débute avec 2 points de destin.

\subsection{La Voie des Origines}

La Voie des Origines permet de définir les traits (ou compétences) du personnage de manière cohérente avec l'univers et de créer un début de background.

Elle se déroule en trois étapes successives.

\subsubsection*{Grandir dans une communauté}

Lorsqu'un bambin évolue dans une communauté, il apprend naturellement au contact des autres. C'est pourquoi le premier trait à +1 est déterminé par l'activité principale de la communauté où il vit. Par exemple, un enfant ayant grandi dans un petit village vivant majoritairement de l'agriculture aura comme premier trait Paysan, tandis que s'il grandit dans un bourg spécialisé dans la briquèterie, il aura Artisan en premier trait.

Si vos joueurs veulent être originaires de la même communauté, ils auront donc ce trait en commun.

\emph{Exemple : Le groupe de joueurs a décidé de créer des personnages originaires de la Colline-aux-Chênes, une communauté dont la spécialité est la charbonnerie. Leur trait d'origine sera donc Forestier. Les joueurs inscrivent donc le trait Forestier avec une valeur de +1.}

\subsubsection*{Quand je serai grand, je ferai comme Papa !}

On apprend toujours plus de ses parents, aussi le métier de l'un des parents apparaîtra en second trait à +1. On fera toutefois attention à sélectionner un trait différent du premier. Ce n'est pas forcément le métier du père, mais peut aussi être celui du grand-père, du tonton taré, ou même d'un mentor de l'enfance.

\emph{Exemple : Bob a décidé que le père de son personnage était membre du conseil du village, et que son personnage a beaucoup appris de celui-ci sur la manière de gérer efficacement une communauté. Il inscrit donc le trait Administrateur avec la valeur de +1.}

\subsubsection*{L'apprentissage final}

Pour finir, le personnage a appris un métier à part, qui fait de lui quelqu'un de spécial dans sa communauté (et le fera sans doute partir sur les routes, le pauvre). Ce trait final a une valeur de +2, et doit être un trait différent des deux premiers.

\emph{Exemple : Bob estime que son personnage a appris le métier des armes et inscrit donc le trait Soldat avec une valeur de +2. Son personnage a donc comme traits Forestier +1, Administrateur +1 et Soldat +2.}

\begin{mdframed}
\textbf{Option : Jouer un personnage plus expérimenté}

Avec l'accord du MJ, il est possible de créer un personnage ayant un peu plus de bouteille. Dans ce cas, le joueur sélectionne un trait supplémentaire à +1, ou ajoute 1 à la valeur d'un de ses traits existants. Le personnage commencera avec un point de destin en moins.
\end{mdframed}

\subsection{Les aspects du personnage}

\subsubsection*{La crainte}

Choisissez une crainte. C'est la principale peur de votre personnage. Soyez cohérent avec l'univers dans lequel vous vivez. La source de cette peur doit exister dans l'univers, et le MJ doit pouvoir l'utiliser à votre encontre.

\subsubsection*{Le réconfort}

À présent, choisissez une source de réconfort. Cela peut être une personne, un objet, un lieu, en gros quelque chose de physique. Attention toutefois, l'avantage donné par la source de réconfort ne fonctionne que lorsque vous êtes physiquement en sa présence.

\subsection{L'équipement}

Décidez maintenant de l'équipement que portera votre personnage. Porte-t-il une armure lourde ? légère ? Avec quel type d'arme combat-il ? Soyez à nouveau cohérent avec l'univers. Pas de lance plasma dans un univers médiéval !

\newpage

\section{Les traits}

\lettrine[lines=3]{\initfamily\textcolor{darkgreen}{U}}{n personnage} est défini par ses caractéristiques mais surtout par ses compétences appelées traits. Les traits sont décrits dans cette section.

Lors d'un jet de dés, on ne peut utiliser qu'un seul trait. Celui-ci est déterminé en concertation entre le MJ et le joueur.

\subsection*{Administrateur}

L'Administrateur sait gérer une communauté ou une entreprise, de son aspect financier à l'aspect humain, en passant par la gestion de ses activités au jour le jour.

\subsection*{Artisan}

L'Artisan peut transformer les matériaux bruts en objets finis, il peut être formé dans le travail du bois, des métaux, ou de toute autre matière. On regroupe sous ce trait la plupart des métiers de production.

\subsection*{Artiste}

L'Artiste crée la beauté dans ce monde en décrépitude. Il peut être peintre, musicien, poète, peu importe...

\subsection*{Athlète}

L'Athlète est à la fois vif et fort. Il peut courir vite, sauter haut, ou esquiver des dangers face auxquels d'autres succomberaient rapidement.

\subsection*{Bagarreur}

Le Bagarreur sait comment se comporter lorsque les choses s'enveniment. Lorsqu'il est confronté à des conflits, il sait comment clore les débats par un poing final.

\subsection*{Batelier}

Le Batelier est particulièrement à l'aise sur les cours d'eau ou les mers tourmentées. Que ce soit sur un immense galion ou une petite barque, il pourra toujours manoeuvrer l'embarcation.

\subsection*{Chasseur}

Le Chasseur sait traquer et abattre toutes sortes de proies et de prédateurs. Il est doué pour se cacher dans les milieux naturels, notamment lorsqu'il est lui-même une proie...

\subsection*{Citadin}

Le Citadin sait se comporter parfaitement en milieu urbain. Il sait y trouver son chemin, parler aux habitants ou encore trouver des opportunités dans les communautés importantes.

\subsection*{Éclaireur}

L'Eclaireur est un excellent observateur. Il sait s'orienter, observer et faire un rapport, tout comme éliminer discrètement une menace un peu trop présente.

\subsection*{Érudit}

L'Erudit est féru de connaissances diverses et variées. Dans de nombreux cas de figure, il saura apporter sa pierre à l'édifice en apportant son savoir.

\subsection*{Ferrailleur}

Le Ferrailleur est capable de récupérer la moindre denrée précieuse dans un tas de déchets. Il est également capable de réparer des objets avec ce qu'il trouve.

\subsection*{Forestier}

Le Forestier est l'expert du sous-bois et de tout milieu naturel. Il sait également comment entretenir une production de bois et s'orienter en dehors des sentiers balisés, tout en gardant une certaine prudence.

\subsection*{Gentilhomme}

Le Gentilhomme sait comment se comporter avec les Grands de ce monde. Il sait évoluer dans les hautes sphères de la société, et en tirer des avantages.

\subsection*{Intrigant}

L'Intrigant est un as de la langue de bois. Il sait obtenir ce qu'il veut de tout un chacun en manipulant les humeurs et les envies.

\subsection*{Larron}

Le Larron a toujours plus d'un tour dans son sac, ou plutôt dans le sac des autres. Il est discret et sait effectuer des emprunts à durée indéterminée.

\subsection*{Marchand}

Le Marchand sait comment fonctionne le commerce, malgré les dangers de ce monde. Il est bon négociateur et est doué pour estimer la valeur des choses.

\subsection*{Médecin}

Le Médecin sait soigner les gens et les animaux. Il a une bonne connaissance des plantes médicinales, ainsi que de l'anatomie.

\subsection*{Orateur}

L'Orateur sait parler. Et bien parler de surcroît ! Il n'a pas le trac devant une foule et saura les convaincre à son point de vue.

\subsection*{Paysan}

Le Paysan sait travailler la terre, cultiver les céréales et légumes, et s'occuper du bétail. Il sait également se comporter dans un environnement rural.

\subsection*{Saltimbanque}

Le Saltimbanque sait comment divertir les masses par ses jeux, ses chants ou ses poèmes. Il est également doué pour convaincre les gens.

\subsection*{Soldat}

Le Soldat sait combattre, commander des troupes et les entraîner. Il sait se servir d'une arme comme l'entretenir, et peut élaborer des stratégies.

\subsection*{Vagabond}

Le Vagabond est capable de survivre dans toutes les conditions les plus extrêmes, en trouvant sa subsistance dans toute situation.

\subsection*{Voyageur}

Le Voyageur connaît les codes des voyages et la topographie des routes. Il sait comment se comporter dans une caravane ou dans les relais.

\newpage

\section{Le combat et les dangers}

\lettrine[lines=3]{\initfamily\textcolor{darkgreen}{I}}{l arrive} que les personnages soient obligés à combattre pour sauver leur misérable existence. Dans cette section, on abordera les règles des combats, des dommages et du soin.

\subsection{Défense passive ou active}

La défense correspond au seuil de difficulté pour blesser un personnage (sur n'importe quelle des trois caractéristiques). Elle peut être soit active, soit passive.

Elle est dite \textbf{active} lorsque le personnage utilise son action du tour pour se défendre uniquement, et \textbf{passive} lorsqu'il cherche à agir lors de son tour.

La défense passive est égale à la \textbf{réserve de la caractéristique attaquée + 5}. Un personnage ayant une caractéristique de 5 en corps aura donc un seuil de difficulté de 10 pour être blessé par une attaque physique.

La défense active est définie par un jet de la caractéristique attaquée, soit la \textbf{réserve de caractéristique + un éventuel trait + 1d12}. En effet, même en se défendant de manière active, il est toujours possible de se tromper et d'ouvrir sa garde.

\subsection{Tours de combat}

Un combat est défini en tours. Un tour se déroule de la manière suivante :

Les personnages-joueurs ont chacun une action, et agissent l'un après l'autre (ici, pas de règle d'initiative, laissez les joueurs s'arranger entre eux). Ils peuvent :

\begin{itemize}
\item Attaquer
\item Se défendre activement
\item Aider un allié
\item Fuir (quitter la confrontation)
\end{itemize}

\subsection{Actions lors du combat}

\subsubsection*{Attaquer}

Une action d'attaque est effectuée par un jet de la caractéristique associée, à savoir le Corps pour les attaques physiques (au corps à corps ou à distance), l'Esprit pour les attaques mentales, et l'Âme pour les attaques sociales, contre la défense associée (active ou passive)

Si le jet est réussi, alors l'attaque inflige les \textbf{dégâts de l'arme + la marge de réussite}.

\subsubsection*{Se défendre activement}

Comme indiqué dans la section sur la défense, lorsqu'un personnage se défend, il utilise sa défense active au lieu de sa défense passive pour définir le seuil de difficulté.

\subsubsection*{Aider un allié}

Un personnage peut choisir d'aider un allié, soit à attaquer, soit à se défendre, soit à fuir un combat. Ce faisant, il doit effectuer un jet de \textbf{la caractéristique associée + un éventuel trait} contre une difficulté de \textbf{12}. En cas de réussite, son allié aura un bonus de \textbf{+3} à son jet.

\subsubsection*{Fuir une confrontation}

Par moments, il est vraiment difficile d'affronter certaines créatures ou personnes. Aussi, il est possible de fuir une confrontation. Pour ce faire, il faut faire un jet de Corps (dans le cas d'un combat), d'Esprit (dans le cas d'une confrontation mentale) ou d'Âme (dans le cas d'une confrontation sociale). La difficulté de ce jet est de 12.

\begin{itemize}
\item Si le jet est \textbf{réussi}, alors le personnage s'est échappé de la confrontation et ne pourra plus y participer à moins qu'il ne décide d'y reprendre part. Cela lui évite ainsi de souffrir de dommages lors de la confrontation.
\item Si le jet est \textbf{une réussite avec complication}, le personnage s'est échappé de la confrontation, mais a subi un point de dommage dans la caractéristique en question lors de sa fuite.
\item Si le jet est un \textbf{échec}, le personnage n'a pas pu fuir et doit donc continuer la confrontation.
\end{itemize}

\subsection{Dommages et récupération}

\subsubsection*{Infliger et encaisser des dégâts}

Lorsqu'un personnage est blessé, il subit des dégâts dans la caractéristique liée. Ces points sont directement impactés dans sa réserve. On déduira toutefois la Protection induite par l'armure éventuelle.

\emph{Par exemple, Bob (Corps 7/7) est en train de combattre un scolopendre géant (Créature 8/8, Morsure 2) qui l'a attaqué pendant la nuit. L'effroyable bestiole l'attaque et réussit à le blesser grâce à un jet d'attaque (1d12 + 8 : 6+8 = 14, ce qui est supérieur à la défense passive de Bob : 12). Sa marge de réussite est de 2 et les dégâts de la morsure sont de 2, pour un total de 4 points de dommages. La réserve de Corps de Bob tombe donc à 3 points. Il vient d'être salement blessé.}

Lors d'une réussite critique, on calcule la marge de réussite + les dommages de l'arme et on multiplie par 2 le total.

\subsubsection*{Récupération}

Afin de récupérer sa réserve de caractéristique, plusieurs moyens sont envisageables.

\begin{itemize}
\item \textbf{Récupération naturelle :} Pour chaque tranche complète de 4h de repos, un personnage regagne un point de réserve. Cette récupération est doublée lorsque le personnage est en présence de sa source de réconfort.
\item \textbf{Soin :} Il est possible de soigner des blessures physiques par un jet d'Esprit + Médecin. La difficulté du jet est égale à 12 + nombre de points à faire récupérer. Les blessures mentales peuvent être soignées par un jet d'Âme + Orateur de la même manière. Il n'est pas possible de soigner les blessures d'Âme de cette manière.
\item \textbf{Fin de scénario :} À la fin d'un scénario, les personnages regagnent toutes leurs réserves jusqu'à leur valeur de caractéristique.
\end{itemize}

\subsubsection*{Pertes définitives de caractéristique}

Certaines blessures peuvent être particulièrement dévastatrices. Il est possible pour certaines armes et/ou créatures d'infliger des dommages irréversibles à un personnage. Lorsque cela arrive, le personnage ne perd pas de points de sa réserve mais directement dans sa valeur de caractéristique. Si la valeur devient inférieure à la réserve, réduisez alors la réserve pour qu'elle soit égale à la valeur.

\newpage

\section{L'équipement}

\lettrine[lines=3]{\initfamily\textcolor{darkgreen}{L}}{e matériel} dépend avant tout de l'univers utilisé. Par exemple, un personnage évoluant dans l'univers forestier d'\emph{Unterwald} n'aura pas forcément accès au même équipement qu'un \emph{Voyageur du Dehors}.

On peut définir trois catégories principales d'équipement, à savoir les armes, les équipements de protection et le reste du matériel.

\subsection{L'armement}

Une arme est définie par trois facteurs :

\begin{itemize}
\item \textbf{Portée :} À quelle distance peut-elle porter (corps-à-corps, proche, loin). Par exemple, un couteau peut être utilisé au corps à corps et à faible distance. Un pistolet peut être utilisé à faible distance (au corps à corps, il compte comme une matraque), etc...
\item \textbf{Taille :} De quelle taille est-elle (petit, moyen, grand). De cela, on définira les dégâts de base de l'arme.
\item \textbf{Attributs :} A-t-elle des attributs spéciaux ? Certaines armes peuvent avoir des attributs spéciaux, comme une grenade qui possède un rayon d'effet.
\end{itemize}

\begin{tabular}{lr}\\\toprule  
Taille de l'arme (Exemples) & Dégâts \\\midrule
Petite (couteau, petit pistolet, matraque) & 1 \\  
Moyenne (épée, gros pistolet, pistolet-mitrailleur) & 2 \\ 
Grande (épée à deux mains, fusil de chasse, fusil de précision) & 3 \\  \bottomrule
\end{tabular}

\subsection{Équipements de protection}

Oui, il faut se protéger aussi. Une armure est définie par deux facteurs: 

\begin{itemize}
\item \textbf{Protection :} La valeur de Protection définit le nombre de points déduits des dommages physiques reçus.
\item \textbf{Attributs :} A-t-elle des attributs spéciaux ? Certaines armures peuvent avoir des attributs spéciaux, comme des colifichets qui apportent une Protection mystique qui déduit les dommages d'Esprit.
\end{itemize}

\begin{tabular}{lr}\\\toprule  
Type d'armure (Exemples) & Protection \\\midrule
Légère (Blouson de moto, armure de cuir) & 1 \\  
Moyenne (Cotte de mailles, Gilet tactique) & 2 \\ 
Lourde (Armure de plaques, Gilet pare-éclats) & 3 \\  \bottomrule
\end{tabular}

\subsection{Equipement divers}

On choisira de ne pas limiter l'équipement (avoir une feuille à côté pour tout noter est recommandé).

Seuls les équipements apportant un bonus devraient être notés sur la feuille principale.

\emph{Par exemple, Bob possède une caisse à outils très complète qui lui accorde un bonus de +2 aux tests de bricolage et de réparation.}