\chapter{Unterwald}
\newpage

\section{À travers le Col Brun}

L’homme lança un regard inquiet dans son dos. Les bruits de pas avaient cessé, mais il n’était toujours pas tranquille. Derrière lui, le soleil commençait à disparaître derrière les montagnes, et bientôt, la nuit allait s’abattre sur la Haute-Forêt.

Cela faisait plusieurs heures déjà qu’il progressait sur la route du Col Brun, il en avait perdu le compte, et l’obscurité l’empêchait déjà de lire les jalons indiquant les distances.

\emph{« Il faut que j’atteigne Villars avant la nuit » marmonnait-il dans son écharpe, « je risque de me paumer… Putain de mercs. Si seulement ils pouvaient se perdre, eux. »}

Trois jours, trois putains de jours que ces mercenaires le poursuivaient. À vrai dire, depuis qu’il avait quitté le Nouveau Bastion muni d’une lettre pour le Comte de Brillance, ils étaient sur ses talons.

Ses mains serraient le tube de cuir contenant la missive. Il fallait qu’elle arrive à son destinataire, coûte que coûte. La survie des habitants de la Basse-Forêt en dépendait.

\emph{« Paul, vous êtes le meilleur de nos messagers. Si cette lettre n’arrive pas à bon port, nous sommes fichus. Si les pirates du Rigg nous attaquent, il ne restera plus rien de nos villes. Plus rien de ce que nous avons mis tant de temps à construire. Souvenez-vous de ce qu’ils ont fait à Lucarn. Nous sommes les prochains. Hâtez-vous, et nous arriverons peut-être à résister. Vous êtes notre dernière chance. »}

Les mots de la Comtesse de la Basse-Forêt résonnaient encore dans l’esprit du messager. Il connaissait les enjeux, et était bien déterminé à accomplir sa mission. La vie des siens était dans la balance. Sa sœur, ses neveux, et tous les habitants de la province. Leurs vies étaient entre ses mains. Littéralement.

Un craquement sinistre retentit sur sa droite. Instinctivement, il tourna la tête dans cette direction avant de se raviser très rapidement. Les créatures qui erraient aux alentours du Col Brun n’étaient jamais agréables à voir… Plus d’un voyageur avait perdu la raison à leur seule vue, et les abords de la route étaient jonchés des restes de ceux qui avaient préféré le vide et l’étreinte de l’oubli à cette vision d’horreur.

Le chemin serpentait toujours entre les arbres, et la pente devenait de plus en plus raide. Le Col Brun n’était plus très éloigné, à présent. Il redoubla d’efforts, sentant son souffle s’amenuiser alors qu’il gravissait la route de montagne qui culminait à plus de cinq cents mètres en amont de la vallée d’où il était parti.

\emph{« Encore un petit effort, mon vieux. Tu peux le faire. Une fois au col, tu pourras souffler. »} murmurait-il, plus pour se persuader que par véritable conviction.

Une ombre apparut à la lisière de son champ de vision. Proche. Beaucoup trop proche. Il porta la main à son pistolet à poudre noire. Pas pour l’ombre. Pour lui. Piochant dans ses dernières réserves, il se mit à courir, alors que retentissait un croassement lugubre dans son dos, puis un deuxième à sa droite… et un troisième à sa gauche. \emph{« Bordel, comme si un ça ne suffisait pas… »}

Son sang se glaça lorsqu’il aperçut l’une des bêtes. C’était une monstruosité écailleuse, à la tête hérissée de pointes acérées, et à la gueule suintant d’un ichor noirâtre entre ses crocs saillants. Une paire d’ailes membraneuses complétait la vision de cauchemar, et Paul sentit sa volonté faillir et sa raison s’évanouir alors qu’il retenait à grand peine un hurlement d’effroi.

Il prit ses jambes à son cou et fuit l’horreur qu’il avait contemplée un instant. Pendant d’interminables minutes, il courut, jusqu'à ce que ses poursuivants ne soient plus qu'un mauvais souvenir. Alors qu'il reprenait son souffle après sa course effrénée, la réalité lui fit l'effet d'un coup de poing dans l'estomac. Il avait rompu la promesse qu'il s'était faite à lui-même lorsqu'il était devenu messager. Il avait violé sa règle la plus élémentaire.

Il avait quitté la route.

\section{Unterwald, en bref}

\section{Atlas d'Unterwald}

\subsection{La Basse-Forêt}

TODO : Insert map

\subsubsection*{Fort de Pilat}

Juché à plus de 2100 mètres d'altitude le fort de Pilat domine la région environnante et est très difficile d'accès aux personnes non entraînées. Le chemin qui y mène est très raide et n'est pas accessible aux chariots au dessus de 1700 mètres. Seuls des guides entraînés peuvent y monter à cheval.

La position culminante du fort sur la montagne du Pilat permet une vue dégagée sur toute la région de la Basse-Forêt, et abrite un corps de forestiers et de guides de montagne bien entraînés.

\subsubsection*{Alp}

Au pied du Pilat, on trouve la petite bourgade d'Alp, forte de 400 habitants et dont la situation dans une vallée irriguée lui permet d'avoir une économie tournée vers l'agriculture.

\subsubsection*{Serment}

Au bord du lac du même nom, on trouve le bourg de Serment, peuplé de 900 habitants et entouré d'une forte muraille de pierre. Ce bourg abrite la garnison du sud de la région.

\subsubsection*{Creux de l'eau}

Au bord du lac du Serment, on découvre le petit village de Creux de l'eau et ses 150 habitants. L'économie est tournée majoritairement vers la pêche et l'exploitation de l'argile.

Il s'agit également du village marquant la frontière de la Basse-Forêt sur la route du Col Brun. Au delà, on trouve la région de la Haute-Forêt.

\subsubsection*{Nouveau-Bastion}

Capitale de-facto de la province de la Basse Forêt, le bourg de Nouveau-Bastion date seulement de 30 ans, lorsqu'une attaque de créatures ailées a frappé la ville de Bastion, renommée dès-lors Vieux-Bastion.

On y trouve la plus grande garnison de la région, et le bourg compte plus de 1200 habitants.

\subsubsection*{L'Envol}

Situé sur une plaine étrangement aplanie, l'Envol est un petit village de 200 habitants tirant son nom d'un curieux oiseau de métal trouvé par un fermier alors qu'il retournait son champ. L'oiseau en question est exposé sur la place centrale du village.

\subsubsection*{Le Tir aux Loups}

Le Tir aux Loups est une communauté forestière de montagne d'environ 100 habitants. Les villageois y sont bourrus, et ont coutume de porter des vêtements cousus à partir de fourrure de loups.

Autrefois, un chemin permettait de se rendre dans les montagnes au sud, mais celui-ci a été barré par un glissement de terrain, et rares sont les inconscients à s'aventurer au delà.

\subsubsection*{Vieux-Bastion}

Vieux-Bastion, autrefois Bastion, est le lieu de l'ancien siège du pouvoir de la Basse-Forêt. On trouve dans ses ruines les échos d'une occupation passée, juste troublée par les animaux sauvage et la nature qui y reprend ses droits.

Des voyageurs avides de richesses explorent parfois les ruines à la recherche d'objets abandonnés lors de la destruction du bourg, mais nul ne sait si les créatures qui jadis attaquèrent ne hantent pas encore les lieux.

\subsubsection*{La Dalle}

Ancien point de passage sur la route du Tir aux Loups, la Dalle était autrefois une communauté paysanne, mais le village fut attaqué par une bande de pillards qui ravagea le village.

\subsubsection*{Fort de la Corne}

Le Fort de la Corne, juché sur la Corne de Bastion, est situé à 1900 mètres d'altitude. Un chemin balisé permet d'y accéder en chariot, et le fort abrite la demeure du Seigneur de la Basse-Forêt.

\subsection{La Haute-Forêt}

TODO : Insert map

TODO : Insert descriptions

\subsection{Le Rigg}

TODO : Insert map

TODO : Insert descriptions

\subsection{Contrées plus lointaines}

\section{Maîtriser Unterwald}

\section{Bestiaire et adversité}

\section{Scénario: Le vol de l'Oiseau de Fer}