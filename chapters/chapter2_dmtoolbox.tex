\chapter{Les outils du meneur de jeu}
\newpage

Si vous êtes joueur, je vous déconseille d'aller plus loin. Vous pourrez toujours découvrir les univers en jeu, ce qui est bien plus agréable !

\section{Concevoir l'adversité}

Dans Don't Leave The Road, l'adversité peut être multiple. On peut trouver des environnements hostiles, des créatures à l'appétit démesuré, des cultistes \emph{(Noooon ! Pas de tentacules ! S'il vous plait !)} fous, des pillards cannibales ou d'autres innombrables dangers pour les habitants de ce(s) monde(s) dévasté(s).

\subsection{Gérer la difficulté d'une action}

Le système s'articulant autour d'un seuil de difficulté à dépasser, voici un tableau récapitulatif des difficultés standard. Bien évidemment, vous pouvez tout à fait utiliser des valeurs intermédiaires, celles présentées ici sont juste à des fins d'exemple.

\begin{tabular}{@{}lll@{}}
\toprule
\textbf{Difficulté} & \textbf{Exemple}                                                                                                                                       & \textbf{Seuil} \\ \midrule
\textbf{Enfantin}   & \textit{\begin{tabular}[c]{@{}l@{}}Trouver du bois de chauffe dans une forêt\\ Se faufiler discrètement derrière un garde endormi\end{tabular}}        & 6              \\\midrule
\textbf{Facile}     & \textit{\begin{tabular}[c]{@{}l@{}}Convaincre un bon ami de vous aider à\\ effectuer une tâche ardue\end{tabular}}                                     & 9              \\\midrule
\textbf{Moyen}      & \textit{\begin{tabular}[c]{@{}l@{}}Combattre un ennemi classique.\\ Se repérer grâce à des points d'observation\\ Résister à sa Crainte.\end{tabular}} & 12             \\\midrule
\textbf{Complexe}   & \textit{Convaincre une personne hostile de vous aider.}                                                                                                & 15             \\\midrule
\textbf{Ardu}       & \textit{\begin{tabular}[c]{@{}l@{}}Se faufiler dans une demeure gardée et éclairée.\\ Trouver de l'eau pure dans un marais putride.\end{tabular}}      & 18             \\ \bottomrule
\end{tabular}

\subsection{Créer des adversaires}

